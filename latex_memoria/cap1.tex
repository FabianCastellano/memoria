% ---------------------------------------------------------------------------------------
\chapter{Introducci\'on}\label{chap1}

En el an\'alisis de datos, es com\'un verse enfrentado con datos problem\'aticos, valores faltantes, datos at\'ipicos, o datos que no distribuyan de forma normal. Para el \'ultimo caso es com\'un aplicar transformaciones estos para que su distribuci\'on se asemeje a una Normal. Una de las transformaciones m\'as comunes para lograr esto es la transformaci\'on de Box-Cox, la cual busca, a trav\'es de un problema de optimizaci\'on, encontrar el valor de $\lambda$ que mejor ajuste los datos a una distribuci\'on Normal. Esta transformaci\'on suele aplicarse sobre  vectores unidimensionales, y no ha sido extendida a matrices $d$-dimensionales en las que existe correlaciones de adyacencia, excepto en una cantidad muy reducida de trabajos, en los que destacan Lee et. al. \textit{(MR Image Segmentation Using a Power Transformation Approach)}\cite{lee2009mr}, y Bicego y Baldo \textit{(Properties of the Box-Cox Transformation for Pattern Classification)}\cite{bicego2016}. 

En el primero se trata el problema de segmentaci\'on de im\'agenes de resonancia magn\'etica, y utlizan la transformaci\'on simplemente aplananado la matriz de la imagen. En el segundo se estudian distintas propiedades de la transformaci\'on de Box-Cox sobre im\'agenes, en particular para clasificaci\'on de patrones. En el segundo trabajo se propone una perspectiva interesante, utilizar el histograma de la imagen para encontrar el valor de $\lambda$ que mejor ajuste los valores de este a una distribuci\'on normal. Dado este intérs por aplicar la transformaci\'on de Box-Cox a im\'agenes, y la falta de trabajos en esta direcci\'on, nos interesa estudiar la relaci\'on que existe entre la imagen original y la imagen transformada, en particular para distintos valores de $\lambda$, y para distintos m\'etodos de obtenci\'on de $\lambda$.

Recordemos que esta transformaci\'on es no lineal, por lo que nos interesa utilizar m\'etodos de comparaci\'on que nos permitan detectar relaciones no lineales entre las variables. Por esto en el trabajo en el Cap\'itulo \ref{chap2}, con dos m\'etodos de comparaci\'on: \textit{Maximal Information Coefficient} o MIC, y \textit{Distance Correlation} o dCor. Revisaremos como se define cada uno, y como podemos calcularlo, adem\'as de algunos ejemplos de su aplicaci\'on.

Continuaremos en el Cap\'itulo \ref{chap3} revisando como trabajamos con im\'agenes, su interpretaci\'on como matrices, y como procederemos a aplicar m\'etodos de comparaci\'on sobre estas im\'agenes. Adem\'as revisaremos el banco de im\'agenes que utilizaremos para realizar los experimentos. En este cap\'itulo tambi\'en discutiremos como podemos aplicar los m\'etodos de correlaci\'on estudiados en el cap\'itulo \ref{chap2} sobre im\'agenes.

Luego, en el Cap\'itulo \ref{chap4} procederemos a estudiar la transformaci\'on de Box-Cox, su definici\'on, como podemos aplicarla sobre im\'agenes, y discutiremos distintos m\'etodos para la obtenci\'on del valor $\lambda$. 

Finalmente en el Cap\'itlo \ref{chap5} procederemos a realizar experimentos num\'ericos, en los que aplicaremos la transformaci\'on de Box-Cox sobre im\'agenes, y compararemos la imagen original con la transformada, estudiaremos tanto la relaci\'on para un rango de valores de $\lambda$, como para distintos m\'etodos de obtenci\'on de $\lambda$.

El objetivo de este trabajo es estudiar la relaci\'on que existe entre la transformaci\'on de Box-Cox y la imagen original, en particular para distintas formas de encontrar valores de $\lambda$, y para distintos tipos de im\'agenes. Adem\'as de proponer un m\'etodo para encontrar el valor de $\lambda$. Junto con esto, se explora el concepto de comparaci\'on de im\'agenes, proponiendo un m\'etodo novedoso para comparar im\'agenes que no se basa en la comparaci\'on de pixeles, sino que en la comparaci\'on de la distribuci\'on de los valores de los pixeles, utilizando el histograma de la imagen como base para la comparaci\'on.