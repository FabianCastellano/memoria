% ---------------------------------------------------------------------------------------
\chapter{Introducci\'on}\label{chap1}

En el an\'alisis de datos, es com\'un verse enfrentado con datos problem\'aticos, valores faltantes, datos at\'ipicos, o datos que no distribuyan de forma normal. En el caso de este \'ultimo, es com\'un aplicar transformaciones a los estos para que su distribuci\'on se asemeje a una Normal. Una de las transformaciones m\'as comunes para lograr esto es la transformaci\'on de Box-Cox, la cual busca a travez de un problema de optimizaci\'on, encontrar el valor de $\lambda$ que mejor ajuste los datos a una distribuci\'on Normal. Esta transformaci\'on suele aplicase sobre  vectores unidimensionales, y no ha sido extendida a matrices $d$-dimensionales en las que existe corrlaciones de adyacencia, excepto en una cantidad muy reducida de trabajos, en los que destacan Lee et. al. \textit{(MR Image Segmentation Using a Power Transformation Approach)}\cite{lee2009mr}, y Bicego y Baldo \textit{(Properties of the Box-Cox Transformation for Pattern Classification)}\cite{bicego2016}. 

En el primero se trata el problema de segmentaci\'on de imagenes de resonancia magn\'etica, y utlizan la transformaci\'on simplemente aplananado la matriz de la imagen. En el segundo se estudian distintas propiedes de la transformaci\'on de Box-Cox sobre ima\'agenes, en particular para clasificaci\'on de patrones. En el segundo trabajo, se propone algo interesante, utilizar el histograma de la imagen para encontrar el valor de $\lambda$ que mejor ajuste los valores de este a una distribuci\'on normal. Dado este interes por aplicar la transformaci\'on de Box-Cox a im\'agenes, y la falta de trabajos en esta direcci\'on, nos interesa estudiar la relaci\'on que existe entre la transformaci\'on de Box-Cox y la im\'agen original de donde proviene,

Recordemos que esta transformaci\'on es no lineal, por lo que nos interesa utilziar m\'etodos de comparaci\'on que nos permitan detectar relaciones no lineales entre las variables. Por esto es que comenzaremos el trabajo en el Cap\'itlo \ref{chap2} dos m\'etodos de comparaci\'on, en particular estudiaremos el \textit{Maximal Information Coefficient} o MIC, y el \textit{Distance Correlation} o dCor, donde revisaremos como se define cada uno, y como podemos calcularlo, adem\'as de algunos ejemplos de su aplicaci\'on.

Continuaremos en el Cap\'itlo \ref{chap3} revisando como trabajamos con im\'agenes, su interpretaci\'on como matrices, y como procederemos a aplicar m\'etodos de comparaci\'on sobre estas imagenes. Adem\'as revisaremos el banco de imagenes que utilizaremos para realizar los experimentos. Luego, en el Cap\'itlo \ref{chap4} procederemos a estudiar la tranformaci\'on de Box-Cox, su definici\'on, como podemos aplicarla sobre im\'agenes, y discutiremos distintos m\'etodos para la obtenci\'on del valor $\lambda$. 

Finalmente en el Cap\'itlo \ref{chap5}, ya con todo esto cubierto, procederemos a realizar experimentos num\'ericos, en los que aplicaremos la transformaci\'on de Box-Cox sobre im\'agenes, y compararemos la im\'agen original con la transformada, utilizando los m\'etodos de comparaci\'on estudiados en el cap\'itulo \ref{chap2}.

El objetivo de este trabajo es estudiar la relaci\'on que existe entre la transformaci\'on de Box-Cox y la im\'agen original de donde proviene, en particular para distintas formas de encontrar valores de $\lambda$, y para distintos tipos de im\'agenes. Adem\'as de proponer un m\'etodo para encontrar el valor de $\lambda$. Junto con esto, tambie\'en se explora el concepto de comparaci\'on de ima\'agenes, proponiendo un m\'etodo novedoso para comparar im\'agenes que no se basa en la comparaci\'on de pixeles, sino que en la comparaci\'on de la distribuci\'on de los valores de los pixeles, utilizando el histograma de la imagen como base para la comparaci\'on.