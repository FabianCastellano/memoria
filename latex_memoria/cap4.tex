% ---------------------------------------------------------------------------------------
\chapter{Otros Métodos de Comparación de Imágenes}\label{chap4}


\section[]{Correlaci\'on local} 

	%chen2010.pdf

	\subsection{Discucion sobre el coef.}

	donde se publico, como su ocupa, despcion en palabras

	La correlaci\'on local, tamb\'ien conodica como coeficiente no param\'etrico de Chen, o coeficiente de Chen. Este, sin realizar supuestos sobre distribuciones, detecta relaciones no lineales al invenstigar un mont\'on de correalciones locales. 

	\subsection{Definiciones}

	La definici\'on del m\'etodo est\'a basada en en el concepto de integrales de correlaci\'on, las cuales se definen de la siguiente forma:
	\begin{defn}
		$$
		I(r)=\lim _{N \rightarrow \infty}\left\{\frac{1}{N^{2}} \sum_{i, j=1}^{N} I\left(\left|z_{i}-z_{j}\right|<r\right)\right\}
		$$
	\end{defn}
	La integral de correlaci\'on cuantifica el el n\'umero promedio de vecinos dentro de un radio $r$. Notemos que esta definici\'on sigue teniendo sentido cuando los datos no son series de tiempo.

	Para desarrollar una medida de asociaci\'on entre vectores, $x$ e $y$, modificamos la definici\'on de $I(r)$ como sigue. Sean $z_i=(x_i,y_i)$ con $i=1,\dots, N$ las observaciones en el conjunto de datos. Sea $|z_i-z_j|$ la distancia euclidiana. Definimos $\hat{I}(r)=\frac{1}{N^{2}} \sum_{i, j=1}^{N} I\left(\left|z_{i}-z_{j}\right|<r\right)$. Las distancias obsevadas son adem\'as linealmente transformadas para que se encuentren entres 0 y 1 antes de calcular $\hat{I}$. Notemos que $\hat{I}$a tiene las propiedades de una funci\'on de distribuci\'on acumulativa. Es no decreciente entre 0 y 1 y continua por la derecha. La funci\'on $\hat{I}(r)$ descrive el patr\'on global de distancias entre vecinos. 

	Nuestro inter\'es principal es la definici\'on  de una metrica para cuantificar la asosiaci\'on no lineal estudiando patrones locale. Dado esto, definimos la densidad de vecinos $D$ de forma similar a la derivada de $\hat{I}$: 

	$$
		\hat{D}(r)= \frac{\vartriangle\hat{I}(r)}{\vartriangle r}
	$$

	Donde $\vartriangle\hat{I}(r)$ denota un cambio en $\hat{I}(r)$. La densidad de vecinos es evaluada en radio distreto r, con $r=0,1/m, 2/m, \dots, 1$  y $m$ es un grosor de malla arbitrario. Una funci\'on de suavizado autom\'atico usando validaci\'on cruzada es usada para elegir un \'optimo el tama\~no $m$ (Vilela et al. 2007) y se aplica para suavizar $D(r)$. En el paper, el tama\~no predeterminado $m$ se establece como $N$, el n\'umero de observaciones y en este trabajo usaremos el mismo $m$. El estad\'istico $\hat{D}$ es una aproximaci\'on discreta de $d\hat{I}(r)/d r$, la cual tiene las propiedades formales de una probabilidad funci\'on de densidad. Por lo tanto, con un ligero abuso de terminolog\'ia nos referimos a $\widehat{D}(r)$ como una distribuci\'on.

	En base a esto definimos la correlaci\'on local. Intuitivamente, las distancias entre los puntos de datos entre dos variables correlacionadas diferir\'ian de las distancias entre dos variables no correlacionadas. Sea $\widehat{D_0}(r)$ la estimaci\'on de una distribuci\'on nula, que se compone de dos vectores sin asociaci\'on. Definimos la correlaci\'on local ($\ell(r)$) como la desviaci\'on de D de la de la distribuci\'on nula a una distancia vecina dada r:

	$$
		\ell(r)=\widehat{D}(r)-\widehat{D}_{0}(r)
	$$

	Este enfoque no asume ninguna distribuci\'on param\'etrica. La flexibilidad de este m\'etodo facilita el cambio de la distribuci\'on nula a cualquier distribuci\'on de inter\'es. 
	
	Por ultimo, definimos el coficiente como de correlaci\'on local m\'axima, o coeficiente de Chen como:
	$$
		M=\max _{r}\{|\ell(r)|\}
	$$
	La interpretaci\'on de $\ell(r)$ como la diferencia de dos distribuciones implica que $M$ puede interpretarse como la distancia bajo la norma del supremo entre $\widehat{D}$ y $\widehat{D_0}$. En otras palabras, definimos el estad\'istico M como la desviaci\'on m\'axima entre dos densidades vecinas subyacentes.


\section[]{Correlaci\'on de Pearson}

	\subsection{Discucion sobre el coef.}
	
		donde se publico, como su ocupa, despcion en palabras
	 
	\section{Definiciones}
	
	El coef. se define como:
	\begin{equation}\label{pearson_orig}
		\rho_{X,Y}=\frac{cov(X,Y)}{\sigma_X\sigma_Y}
	\end{equation}
	
	Para una muestra de tama\~no $N$, tenemos:
	
	\begin{equation}\label{pearson_r}
		r=\frac{\sum_{i}^N\left(x_{i}-\bar{x}\right)\left(y_{i}-\bar{y}\right)}{\sqrt{\sum_{i}^n\left(x_{i}-\bar{x}\right)^{2}} \sqrt{\sum_{i}^n\left(y_{i}-\bar{y}\right)^{2}}}
	\end{equation}
	
	Con $x_i,y_i$ elementos de la muestra y $\bar{x},\bar{y}$ sus respectivos promedios.
	
	Hablar de The Ineffectiveness of the Correlation Coefficient for Image Comparisons
	
	\newpage